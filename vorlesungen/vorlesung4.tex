\vorlesung{4}{Thema der Vorlesung}{03.05.2022}

\begin{defn}{partiell berechenbare Funktionen}
    Für Alphabete $\Sigma, \Gamma$ und eine partielle Fuktion $\Phi : \Sigma^* \rightsquigarrow 
    \Gamma$ ist $Phi$ partiell berechenbar, wenn es ein $k \ in \mathbb{N}$ und eine k-DTM $M$ gibt $\Phi_M = \Phi$. \\

    Ist $\Phi$ total und partiell berechenbar, so ist $\Phi$ berechenbar. \\
    %Wdh total -- jede Eingabe terminiert

    Wir schreiben $RF$ für die Klasse der partiell berechenbaren partiellen Funktionen

    Ist $\Phi : \mathbb{N}_0 \rightarrow \mathbb{N}$, so nennen wir $\Phi$ berechenbar,
    falls $bin \circ \Phi \circ bin^-1$ berechenbar ist.
\end{defn}

\begin{defn}{charakteristische Funktion}
    Sei $L$ eine Sprache über dem Alphabet $\Sigma$. \\
    \begin{itemize}
        \item Die charakteristische Funktion von $L$ als Sprache über $\Sigma$ ist Die
              Funktion $1_L : \Sigma^* \rightarrow \{0,1\}$ mit $1_L(w) = 1$ $\forall w \in L$
              und $1_L(w) = 0$ $\forall w \in \Sigma^* \backslash L$.
        \item Die partielle funktion von $L$ als Sprache über $\Sigma$ ist die partielle Funktion
              $X_L : \Sigma^* \leadsto \{1\}$ mit $X_L(w) = 1$ $\forall w \in L$ und $X_L = \uparrow$ $\forall w \in \Sigma^* \backslash L$
    \end{itemize}
\end{defn}

\begin{bem}
    Sei $L$ eine Sprache über $\Sigma$\\
    \begin{itemize}
        \item $L$ ist genau dann entscheidbar, wenn die char. Funktion von $L$ berechenbar ist.
        \item $L$ ist genau dann rekursivaufzählbar, wenn die partiell charakteristische Funktion von $L$
              partiell berechenbar ist.
    \end{itemize}
\end{bem}

\textbf{Ziel}: Reduzierung der Komplexität verschiedener Parameter von Turingmaschinen. \\
\textbf{Zweck}: Besser handhabar für Beweise von formalen Aussagen. \\

\begin{defn}{normiert}
    Eine TM $M = (Q, \Sigma, \Gamma, \Delta, s, F)$ heißt \textbf{normiert}, wenn $Q = \{0,1,...,n\} = [n]$ für ein
    $n \in \mathbb{N}_0, \Sigma = \{0,1\}, \Gamma = \{0,1,\square\}, s=0, F = \{\{0\}\}$.
\end{defn}

TM lassen sich in normierte deterministische TM umwandeln \\
Die wesentlichen Schritte sind wie folgt: \\
Alle TM mit $\Sigma=\{0,1\}$ und Ausgabe in $\{0,1\}$ lassen sich in det. TM umwandeln,
die die gleiche Sprache erkennen und die gleiche partielle Funktion berechnen. \\

1. Nicht-det. TM  lassen sich in det. TM umwandeln indem wir die verschiedenen Wahlen innerhalb 
unserer Rechnung "nacheinander" durchgehen (Breitensuche). \\

%% Abbildung 2

2. Reduktion von $k$ Bändern auf ein Band mittels Spurentechnik. \\

%% Abbildung 3

3. Das Bandalphabet lässt sich durch $\{0,1,\square\}$ ersetzen, indem man mehrere nebeneinander
liegende Felder für ein einzelnes benutzt. Da die TM immer nur ein Feld liest, muss das gelesene Symbol sich im Zustand gemerkt werden.

%% Abbildung 4

\begin{bem}
    Sei $L = \{0,1\}$ eine Sprache und $\Phi : \{0,1\} \leadsto \{0,1\}^*$ eine partielle Funktion
    \begin{itemize}
        \item $L$ ist genau dann berechenbar, wenn $L$ akzeptierte Sprache einer totalen normierten DTM ist.
        \item $L$ ist genau dann rekursivaufzählbar, wenn $L$ akzeptierte Sprache einer normierten DTM ist.
        \item $\Phi$ ist genau dann partiell berechenbar, wenn $\Phi$ berechnete partielle Funktion einer normierten DTM ist.
    \end{itemize}
    $\Rightarrow$ Berechenbarkeitsbegriff ist unabhängig von Details einer TM. \\
    
    Tatsächlich geht man noch weiter: \textbf{Church-Turing These}: \\
    Berechbarkeit auf einer TM entspricht intuitiver Berechbarkeit. (Achtung: intiuitve Berechenbarkeit ist kein formaler Begriff) \\
    Oft belassen wir es bei intuitiver relativ groben Beschreibungen von TM.
\end{bem}


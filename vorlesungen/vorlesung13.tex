\vorlesung{13}{Formale Grammatiken}{07.06.2022}

\section{Formale Grammatiken}

\textbf{Ziel:}\\
Sprachen nicht über Berechnungsmodelle bilden, sondern direkt.\\
\\
\textbf{Beispiel:}\\
$L = \{0\}^*$\\
$S \to S0 \to S0000 \to 0000$

\begin{defn}{Grammatik}
Eine Grammatik ist ein Tupel $G = (N, T, P, S)$. Dabei ist
\begin{itemize}
    \item $N$ das Alphabet der Nichtterminalsymbole/Variablen
    \item $T$ das Alphabet der Terminalsymbole mit $N \cap T = \emptyset$
    \item $P \subseteq ((N \cup T)^* \setminus T^*) \times (N \cup T)^*$ eine endliche Menge von Regeln, wobei wir für das Paar $(u, v) \in P$ auch $u \to v$ schreiben
    \item $S \in N$ das Startsymbol
\end{itemize}
Eine \textbf{Satzform} von G ist ein Wort $s \in (N \cup T)^*$ und ein Terminalwort von $G$ ist ein Wort $t \in T^*$
\end{defn}

\begin{defn}{Ableitung}
Sei $G = (N, T, P, S)$ eine Grammatik. Eine Satzform $w'$ von $G$ ist in einem Schritt von der Satzform $w$ ableitbar, wenn es Satzformen $u, v, x, y$ gibt, so dass
$$ w = xuy, u \to v \in P \text{ und } w' = xvy $$
gelte. Es bezeichne $\to_G$ die Relation auf der Menge der Satzformen von $G$, so dass $w \to_G w'$ genau dann für Satzformen $w, w'$ von $G$ gilt, wenn $w'$ in einem Schritt aus $w$ ableitbar ist.
Für eine Satzform $w$ von $G$ und ein Nichtterminalsymbol $X \in N$ ist eine \textit{Ableitung} von $w$ aus $X$ in $G$ eine Folge $X = w_1, ..., w_n = w$ mit $w_i \to_G w_{i+1} \forall i \in [n-1]$ und eine Ableitung von $w$ in $G$ ist eine Ableitung von $w$ eines $S$.
\end{defn}

\begin{defn}{Erzeugte Sprache}
Sei $G = (N, T, P, S)$ eine Grammatik. Die von $G$ \textbf{erzeugte Sprache} $L(G)$ ist die Menge der Wörter $w \in T^*$ für die es eine Ableitung von $w$ in $G$ gibt.
\end{defn}

\begin{satz}{}
Eine Sprache ist genau dann rekursiv aufzählbar, wenn sie von einer Grammatik erzeugt wird.\\
\\
\textbf{Beweisidee:}
Gegeben eine Grammatik, ist es halbwegs einfach eine TM anzugeben, die die Ableitungen der Grammatik erzeugt.\\
Gegeben eine det. 1-TM kann man ähnlich zu unserem Vorgehen beim Postschen Korrespondenzproblem das Verhalten der TM mit nur endlich vielen Regeln simulieren.
\end{satz}

\begin{defn}{rechtslinear}
Eine Grammatik $G = (N, T, P, S)$ ist \textbf{rechtslinear}, wenn alle Regeln von $G$ der Form
$$ X \to uY \text{ oder } X \to v $$
mit $X, Y \in N$ und $u \in T^*$ sind.
\end{defn}

\begin{satz}{}
Eine Sprache ist genau dann regulär, wenn sie von einer rechtslinearen Grammatik erzeugt wird.\\
\\
\textbf{Beweisidee:} Zunächst überzeugt man sich, dass eine Sprache $L$ genau dann von einer rechtslinearen Grammatik erzeugt wird, wenn sie von einer Grammatik $G = (N, T, P, S)$ erzeugt wird, bei der alle Regeln von der Form
$$ X \to aY \text{ oder } X \to \lambda $$
mit $X, Y \in N$ und $a \in T$ sind.\\
\\
Eine solche Grammatik wird als \textbf{Grammatik in Simulationsform} bezeichnet. Die Sprache $L$ ist dann erkannte Sprache des endlichen Automaten
$$ A = (N, T, \Delta, S, \{X \in N : X \to \lambda \in P \})$$
mit 
$$ \Delta = \{ (X, a, Y) \in N \times T \times N : X \to aY \in P \} $$
\\
\textbf{Bemerkung:} Wir lassen hier auch EA zu, die stecken bleiben dürfen, d.h. es kann Zustände geben, wo man beim Einlesen eines Symbols nicht weiß, in welchen Zustand der EA wechselt.
\end{satz}


\begin{exam}
Die Sprache $\{0\}^*$ wir von der rechtslinearen Grammatik
$$ G = (\{S\}, \{0, 1\}^*, \{S \to \lambda, S \to 0S\}, S) $$
erzeugt und vom EA
$$ A = (\{S\}, \{0, 1\}^*, \{(S, 0, S)\}, S, \{S\}) $$
erkannt.
\end{exam}

\begin{defn}{kontextfrei}
Eine Grammatik $G = (N, T, P, S)$ ist \textit{kontextfrei}, wenn alle Regeln von $G$ von der Form $ X \to w $ mit $X \in N$ und $w \in (N \cup T)^*$ sind. Eine Sprache heißt kontextfrei, wenn sie von einer kontextfreien Grammatik erzeugt wird. Die Menge aller kontextfreien Sprachen bezeichnen wir mit $CF$.
\end{defn}

\begin{bem}
Die Konstruktion von Kellerautomaten folgt den folgenden Ideen:
\begin{itemize}
    \item Es handelt sich im wesentlichen um Automaten, die nicht wie TM zusätzlich ein Band zur freien Verfügung haben, sondern stattdessen einen Stack/Keller.
    \item Zustandsübergänge hängen vom Zustand des Kellerautomaten, vom eingelesenen Symbol und vom Symbol oben auf dem Stack ab.
    \item Bei jedem Schritt wird das oberste Kellersymbol entfernt und es werden nacheinander die Symbole eines Wortes über dem Kelleralphabet oben auf den Stack gelegt.
\end{itemize}
\end{bem}

\begin{satz}{}
Eine Sprache wird genau dann von einem Kellerautomaten erkannt, wenn sie kontextfrei ist.
\end{satz}

\begin{exam}
Die Sprache $\{0^n, 1^n : n \in \N \}$ wird von der kontextfreien Grammatik
$$ G = (\{S\}, \{0, 1\}^*, \{S \to 0S1, S \to \lambda \}, S) $$
erzeugt.\\
Ein Kellerautomat könnte diese Sprache erkennen, indem er zusätzlich dazu, dass nur Eingaben der Form $0^m 1^n$ akzeptiert werden, für jede $0$ eine $0$ auf den Stack legt und für jede eingelesene $1$ eine $0$ auf dem Stack wieder löscht.
\end{exam}

\vorlesung{3}{Turingmaschinen}{26.04.2022}

Es bezeichne $\rightarrow_M$ die Relation auf der Menge der Konfiguration von $M$, 
sodass $C \rightarrow_M C'$ genau dann für zewi Konfigurationen von $M$ gilt wenn $C'$ nachfolgekonfiguration von $C$ ist. 

\begin{defn}{Rechnung}
    Sei $M = (Q, \Sigma, \Gamma, \Delta, s, F)$ eine k-TM. \\
    
    Eine \textbf{endliche partielle Rechnung} von $M$ ist eine endliche Folge
    $C_1...C_n$ von Konfigurationen von $M$ mit $C_i \rightarrow_M C_{i+1}$ für alle $i \in [n-1]$ \\

    Eine \textbf{unendliche partielle Rechnung} ist eine unendliche Folge $C_1,C_2,...$ von Konfigurationen von $M$
    mit $C_i \rightarrow_M C_{i+1}$ für alle $i \in \mathbb{N}$ \\

    Eine Rechnung von $M$ zur Eingabe $(w_1,...,w_n) \in (\Sigma^*)^M$, wobei $n \in \mathbb{N}$ ist eine endliche partielle Rechnung. \\
    $\text{start}_M = C_1$, $C_2,...C_n$ bei der $C_n$ eine Stoppkonfiguation ist oder eine unendliche partielle Rechnung $\text{start}_M = C_1$, $C_2...$ 
\end{defn}

\begin{bem}
    Ist $M = (Q, \Sigma, \Gamma, \Delta, s, F)$ eine k-DTM, so gibt es für alle $n \in \mathbb{N}$ und $(w_1,...,w_n)$
    genau eine Rechnung zur Eingabe $(w_1,...,w_n)$
\end{bem}

\begin{defn}{Total}
    Eine k-DTM $M = (Q, \Sigma, \Gamma, \Delta, s, F)$ terminiert bei Eingabe $(w_1,...,w_n) \in (\Sigma^*)^n$,
    wobei $n \in \mathbb{N}$, wenn die Rechnung von $M$ zu Eingabe $(w_1,...,w_n)$ endlich ist.
\end{defn}
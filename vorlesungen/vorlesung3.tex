\vorlesung{3}{Turingmaschinen}{26.04.2022}

Es bezeichne $\rightarrow_M$ die Relation auf der Menge der Konfiguration von $M$, 
sodass $C \rightarrow_M C'$ genau dann für zwei Konfigurationen von $M$ gilt wenn $C'$ nachfolgekonfiguration von $C$ ist. 

\begin{defn}{Rechnung}
    Sei $M = (Q, \Sigma, \Gamma, \Delta, s, F)$ eine k-TM. \\
    
    Eine \textbf{endliche partielle Rechnung} von $M$ ist eine endliche Folge
    $C_1...C_n$ von Konfigurationen von $M$ mit $C_i \rightarrow_M C_{i+1}$ für alle $i \in [n-1]$ \\

    Eine \textbf{unendliche partielle Rechnung} ist eine unendliche Folge $C_1,C_2,...$ von Konfigurationen von $M$
    mit $C_i \rightarrow_M C_{i+1}$ für alle $i \in \mathbb{N}$ \\

    Eine Rechnung von $M$ zur Eingabe $(w_1,...,w_n) \in (\Sigma^*)^M$, wobei $n \in \mathbb{N}$ ist eine endliche partielle Rechnung. \\
    $\text{start}_M = C_1$, $C_2,...C_n$ bei der $C_n$ eine Stoppkonfiguration ist oder eine unendliche partielle Rechnung $\text{start}_M = C_1$, $C_2...$ 
\end{defn}

\begin{bem}
    Ist $M = (Q, \Sigma, \Gamma, \Delta, s, F)$ eine k-DTM, so gibt es für alle $n \in \mathbb{N}$ und $(w_1,...,w_n)$
    genau eine Rechnung zur Eingabe $(w_1,...,w_n)$
\end{bem}

\begin{defn}{Total}
    Eine k-DTM $M = (Q, \Sigma, \Gamma, \Delta, s, F)$ terminiert bei Eingabe $(w_1,...,w_n) \in (\Sigma^*)^n$,
    wobei $n \in \mathbb{N}$, wenn die Rechnung von $M$ zu Eingabe $(w_1,...,w_n)$ endlich ist. \\

    Eine k-TM $M = (Q, \Sigma, \Gamma, \Delta, s, F)$ ist total,  wenn für alle $n \in \mathbb{N}$ und 
    $(w_1,...,w_n) \ in (\Sigma^*)^n$ alle Rechnungen von $M$ zur Eingabe $(w_1,...,w_n)$ endlich sind.
\end{defn}

\begin{defn}{akzeptierte Sprache}
    Sei $M = (Q, \Sigma, \Gamma, \Delta, s, F)$ eine k-TM. \\
    Eine Stoppkonfiguration $(q,w_1,...,w_k,p_1,...,p_k)$ ist akzeptierend, wenn $q \in F$ gilt. \\

    Die akzeptierte Sprache $L(M)$ von $M$ ist die Sprache über $\Sigma$, sodass für alle $w \in \Sigma^*$ genau dann $w \in L(M)$ gilt,
    wenn es eine endliche Rechnung $C_1,...,C_N$ von $M$ zur Eingabe $w$ gibt, ber der $C_n$ eine akzeptierte Stoppkonfigruration ist.
\end{defn}

\begin{defn}{entscheidbar}
    Eine Sprache $L$ ist entscheidbar, wenn es eine \textbf{totale} k-TM mit akzeptierter Sprache $L$ gibt.\\
    Wir schreiben $REC$ für die Klasse der entscheidbaren Sprachen. 
\end{defn}

\begin{defn}{rekursivaufzählbar}
    Eine Sprache $L$ ist rekursiv aufzählbar, wenn es eine k-TM mit akzeptierter Sprache $L$ gibt. \\
    Wir schreiben $RE$ für die Klasse der aufzählbaren Sprachen.
\end{defn}

\begin{bem}
    Jede entscheidbare Sprache ist rekursiv aufzählbar\\
    Beispiele: $L=\{\lambda\}$ $\rightarrow$ immer nein; \textbf{alle endliche Sprachen}\\
\end{bem}

\begin{bem}
    Alle endlichen Sprachen sind entscheidbar.\\
    Beispiele:
    \begin{enumerate}
        \item $L \subseteq \Sigma^*$
        \item M-KTM mit $L(M)=L$ \\
              M'-KTM mit $L(M') = \Sigma^* \backslash L$
    \end{enumerate}
\end{bem}

\begin{bem}
    Eine Sprache $L$ über einem Alphabet $\Sigma$ ist genau dann entscheidbar, wenn $L$ und $\Sigma^*\backslash L$ rekursiv aufzählbar sind.
\end{bem}

\begin{defn}{Ausgabe}
    Sei $M = (Q, \Sigma, \Gamma, \Delta, s, F)$ eine k-TM und $C=(q,w_1,...,w_k,p_1,...,p_k)$ eine Konfiguration von $M$. \\

    Die Ausgabe $out_m(C)$ von $M$ bei Konfiguration $C$ ist das Präfix $w \praefix w_1(p_1)...w_k(|w_k|)$ maximaler Länge mit $w \in (\Gamma \backslash \{\square\})^*$
\end{defn}

\begin{defn}{berechnete Funktion}
    Sei $M = (Q, \Sigma, \Gamma, \Delta, s, F)$ eine k-DTM und $n \in \mathbb{N}$. \\

    Die von $M$ berechnete $n$-äre Funktion $\Phi_M$ ist die partielle Funktion \\
    $\Phi_M : (\Sigma^*)^n \leadsto (\Gamma\backslash\{\square\})^*$,
    sodass für alle $(w_1,...,w_n) \in (\Sigma^*)^n$ folgendes gilt: \\

    \begin{itemize}
        \item Ist die Rechnung von $M$ zur Eingabe $(w_1,...,w_n)$ die endliche Rechnung $C_1,...,C_n$ so gilt $\Phi_M(w_1,...,w_n)=out_{M'}(C_M)$
        \item Ist die Rechnung von $M$ zur Eingabe $(w_1,...,w_n)$ unendlich, so gilt $\Phi_M(w_1,...,w_n)\uparrow$
    \end{itemize}

    \textbf{Konvention}: Für $(w_1,...,w_n) \in \Sigma^*$ schreiben wir statt $\Phi(w_1,...,w_n)$ auch $M(w_1,...,w_n)$.
\end{defn}
\vorlesung{10}{Endliche Automaten}{24.05.2022}

\section{Endliche Automaten}
 
\textbf{Ziel:}\\
TM stark einschränken. \\
"keine Bänder" (geht nicht ganz wegen Eingabe)\\

\textbf{Genauer:}
\begin{itemize}
    \item nur ein Band
    \item Bei jedem Rechnungsschritt nur Bewegung nach rechts (Konvention alle Zeichen bleiben bestehen) + Zustandswechsel
    \item Beim Einlesen des ersten $\square$ Symbols muss die Rechnung der Maschine enden. (und sie darf auch nicht vorher enden)
\end{itemize} 

Wir betrachten TM der Form $M = (Q, \Sigma, \Sigma \cup \{\square\}, \Delta, s, F)$ wobei
die Instruktionen in $\Delta$ von der Form $(q,a,q',a',R)$ sind. \\
$\Rightarrow$ Starke Vereinfachung (rechtfertigt neuen Namen)

\begin{defn}{endlicher Automat}
    Ein \textbf{endlicher Automat}, kurz EA, ist ein Tupel $A = (Q, \Sigma, \Delta, s,F)$.
    Dabei ist
    \begin{itemize}
        \item $Q$ eine endl. Menge; Zustandsmenge
        \item $\Sigma$ das Eingabealphabet
        \item $\Delta \subseteq Q \times \Sigma \times Q$ die Übergangsrelation, 
              sodass es für jedes $q \in Q$ und $a \in \Sigma$ ein $q' \in Q$ gibt mit 
              $(q,a,q') \in \Delta$
        \item $s \in Q$ der Startzustand
        \item $F \subseteq Q$ die Menge der akzeptierten Zustände
    \end{itemize}

    Der endliche Automat $A$ ist ein \textbf{deterministischer endl. Automat}, kurz DEA,
    wenn es für alle $(q,a) \in Q \times \Sigma$ genau ein $q' \in Q$ gibt mit $(q,a,q') \in A$ \\

    Im Sinne der obigen Betrachtung entspricht ein EA $A=(Q, \Sigma, \Delta, s, F)$ der 1-TM \\
    $M_A = (Q, \Sigma, \Sigma \cup \{\square\},\{q,a,q',a,R:(q,a,q') \in \Delta\},s,F)$.
\end{defn}

\begin{defn}{Übergangsfunktion eines EA}
    Sei $A=(Q, \Sigma, \Delta, s,F)$ ein EA. Die \textbf{Übergangsfunktion} von $A$ ist die Funktion
    $\delta_A : Q \times \Sigma \rightarrow 2^Q$ mit $\delta_A(q,a) = \{q' \in Q : (q,a,q') \} \in \Delta \} \, \forall q \, \forall a$ \\

    Die \textbf{erweiterte Übergangsfunktion} von $A$ ist die Funktion $\delta_A^* : Q \times \Sigma^* \rightarrow 2^Q$\\
    mit $\delta^*_A(q,\lambda) = \{q\}$ und $\delta_A^*(q,aw)= \bigcup\limits_{q' \in \delta_A(q,a)} \delta_A^* (q',w) \, \forall q \in Q, a \in \Sigma$ 
    und $w \in \Sigma^*$. \\

    Für $Q_0 \subseteq Q$ und $w \in \Sigma^*$ schreiben wir auch $\delta_A^*(Q_0,w)$ statt $\bigcup\limits_{q' \in Q_0} \delta_A^*(q,w)$. \\

    Für einen EA $A = (Q,\Sigma, \Delta, s, F)$ mit entsprechender TM $M_A = (Q, \Sigma, \Gamma, \Delta',s,F)$ und $q \in Q$ und $w \in \Sigma^*$ ist
    $\delta_A(q,w)$ die Menge der Zustände, die sich als erste Komposition der letzten Konfiguration einer Rechnung einer TM = $(Q, \Sigma, \Gamma, \Delta',q,F)$ zu Eingabe $w$
    ergeben. Insbesondere ist $\delta_A^*(s,w)$ die Menge der Zustände, die sich als erste Komposition der letzten Konfiguration einer Rechnung von $M_A$ bei 
    Eingabe $w$ ergeben.
\end{defn}

\begin{bem}
    Sei $A = (Q, \Sigma, \Delta, s, F)$ ein EA
    \begin{enumerate}
        \item $\forall q \in Q, a \in \Sigma$ gilt $\delta^*_A(q,a) = \delta_A(q,a)$.
        \item Ist $A$ ein DEA, $q \in Q, a \in \Sigma, w \in \Sigma^*$, so gilt  $|\delta_A(q,a)| = 1$ und $|\delta^*_A(q,w) =1$.
        \item Seien $u,v \in \Sigma^*$. $\forall q \in Q$ gilt $\delta^*_A(q,uv)=\delta^*_A(\delta^*_A(q,u),v)$ 
              und $\forall \, Q_0 \subseteq Q$ gilt $\delta^*_A(Q_0,uv) = \delta^*_A(\delta^*_A(Q_0,u),v)$.
    \end{enumerate}
\end{bem}

\begin{defn}{Übergangsfunktion eines DEA}
    Sei $A = (Q,\Sigma,\Delta,s,F)$ ein DEA. \\
    
    Die \textbf{Übergangsfunktion} von $A$ ist die Funktion $\delta_{det,A}: Q \times \Sigma \rightarrow Q$ mit
    $\delta^*_A(q,a) = \{\delta^*_{det,A}(q,a)\} \, \forall q \forall a$ \\

    Die \textbf{erweiterte Übergangsfunktion} von $A$ ist die Funktion $\delta^*_{det,A} : Q \times \Sigma^* \rightarrow Q$
    mit $\delta^*_A(q,w) = \{\delta^*_{det,A}(q,w)\} \, \forall q \in Q \, \forall w \in \Sigma$ \\

    Für $Q_0 \subseteq Q$ und $w \in \Sigma^*$ schreiben wir auch $\delta^*_A(Q_0,w)$ statt $\bigcup\limits_{q \in Q_o}\{\delta^*_A(q,w)\} = \delta^*_A(Q_0,w)$.
\end{defn}

\begin{bem}
    Ist $A$ ein DEA, so gelten Bemerkung 5.3 (i) und (iii) auch wenn $\delta_A$ durch $\delta_{det,A}$ und $\delta^*_{det,A}$ ersetzt wird.
\end{bem}

\begin{bem}
    Sie Q eine endliche Menge, $\Sigma$ ein Alphabet, $s \in Q$ und $F \subseteq Q$. \\
    \begin{enumerate}
        \item Für alle Funktionen $\delta : Q \times \Sigma \rightarrow 2^Q$ gibt es genau einen EA $A = (Q,\Sigma,\Delta,s,F)$ mit $\delta_A = \delta$.
        \item Für alle Funktionen $\delta : Q \times \Sigma \rightarrow Q$ gibt es genau einen DEA $A =(Q,\Sigma, \Delta, s,F)$ mit $\delta_{det,A} = \delta$.
    \end{enumerate}
\end{bem}

\begin{defn}{akzeptierte Sprache}
    Sei $A = (Q,\Sigma, \Delta,s,F)$ ein EA. Die Sprache $L(A) := \{w \in \Sigma^* : \delta^*_A(s,w) \cap F \neq \emptyset\}$ ist die \textbf{akzeptierte Sprache} von $A$.
\end{defn}

\begin{defn}{regulär}
    Eine Sprache $L$ heißt \textbf{regulär}, wenn es einen EA $A$ mit $L(A) = L$ gibt. Wir schreiben REG für die Menge aller regulären Sprachen. \\

    Da EA viel einfacher sind als TM, kann man leicht visuelle Darstellungen finden.
\end{defn}

\begin{exam}
    Sei $A := (\{q_0,q_1\},\{0,q\},\Delta,q_0,\{q1\})$ der EA mit \\
    $\Delta = \{(q_0,0,q_0),(q_0,1,q_1),(q_1,0,q_1),(q_1,1,q_0)\}$. \\

    %% Hier ist eine Abbildung des Automaten

    $L(A) = \{w \in \Sigma^* : \#_1(w) = 1 \text{ mod } 2\}$. \\

    \textbf{Erinnerung:} Nichtdet. TM sind nicht mächtiger als TM bezüglich der Menge an Sprachen, die sie entscheiden können.\\


    $\leadsto$ Ähnliches gilt für EA.
\end{exam}

\begin{defn}{Potenzautomat}
        Sei $A = (Q,\Sigma,\Delta,s,F)$ ein EA. \\

        Der \textbf{Potenzautomat} $P_A$ von $A$ ist der DEA \\
        $P_A = (2^Q, \Sigma, \Delta', \{s\},\{P \subseteq P \cap F \neq \emptyset\})$ \\
        mit $\delta_{det,P_A}(Q_0,a) = \bigcup\limits_{q \in Q_0} \delta_A(q,a) \, \forall Q_0 \subseteq Q, a \in \Sigma$.
\end{defn}

\begin{satz}{}
    Eine Sprache $L$ ist genau dann regulär, wenn es einen DEA $A$ gibt mit $L(A)=L$. \\

    \textbf{Beweis:} Sei $A = (Q,\Sigma,\Delta,s,F)$ ein EA mit Potenzautomat $P_A$. \\
    Es genügt zu zeigen, dass $L(A) = L(P_A)$ gilt. \\

    Hierfür genügt es zu zeigen, dass \\
    $\delta^*_{det,P_A}(\{s\},w) = \delta^*_A(\{s\},w)$. $ \color{red}{(*)}$ \\

    Gilt $\forall w \in \Sigma^*$, denn damit folgt
    \begin{align*}
        w \in L(P_A) &\Leftrightarrow \sigma^*_{P_A}(\{s\},w) \cap \{P \subseteq Q : P \cap F \neq \emptyset\} \neq \emptyset \\
        &\Leftrightarrow \delta^*_{det,P_A}(\{s\},w) \cap F \neq \emptyset \\
        &\Leftrightarrow \delta^*_A(s,w) \cap F \neq \emptyset \\
        &\Leftrightarrow w \in L(A).
    \end{align*}

    Wir zeigen $\color{red}{(*)}$ mittels vollständiger Induktion über $|w|$. \\

    Es gilt $\delta^*_{det,P_A}(\{s\},\lambda) = \{s\} = \delta^*_A(s,\lambda).$ \\

    Sei nun $w \in \Sigma^+$ mit $\delta^*_{det,P_A}(\{s\},v) = \delta^*_A(s,v) \forall v \in \Sigma^{\leq|w|-1}$ \\

    Sei $v := w(1)w(2),...,w(|w|-1)$ und $a := w(|w|)$ \\

    Dann gilt: 
    \begin{align*}
        \delta^*_{det,P_A}(\{s\},w) &= \delta^*_{det,P_A}(\delta^*_{det,P_A}(\{s\},v),a) \\
        &= \delta_{det,P_A}(\delta^*_A(s,v),a) \\
        &= \bigcup\limits_{q \in \delta^*_A(s,v)}\delta_A(q,a) \\
        &= \delta^*_A(\delta^*_A(s,v),a) \\
        &= \delta^*_A(s,w) \; \square
    \end{align*}
\end{satz}
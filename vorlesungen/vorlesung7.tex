\vorlesung{7}{Entscheidbarkeit Postsches Korrespondenzproblem}{14.05.2022}

\textbf{Ziel der heutigen Vorlesung}: Zeigen, das das Postsche Korresponzendproblem über dem Binäralphabet entscheidbar ist. 
$H_{init} \leqq PCP_{\{0,1\}}$ \\

Soll aus: \\
$u(1)...(u(|u|))...v(1)...v(|v|)$ \\
$u(1)...(u(|u|))...v(1)...v(|v|)...w(1)...w(|w|)$ \\

eine Lösung werden, so muss nun $v$ oben erzeugt werden. Bei wird unten ein neues Suffix $w$ erzeugt.
Geeignete Wahl der Paare/Steine erlaubt es $w$ zu kontrollieren. Werden Konfigurationen einer TM durch Wörter
dargestellt, so kann erzwungen werden, dass auf eine Darstellung einer Konfiguration die der Nachfolgekonfiguratation
folgen muss. So kann erreicht werden, dass die konstruierte Instanz genau dann lösbar ist, wenn $M$
bei Eingabe $\lambda$ terminiert. \\

Um ein größeres Alphabet zur Verfügung zu haben:

Zeige: $H_{init} \leqq_m MPCP_\Sigma \leqq_m PCP_\Sigma \leqq_m PCP_{\{0,1\}}$

%Lemma 4.15 -- gegenchecken
\begin{lemma}{}
    Für Alphabete $\Sigma$ und $\Gamma$ mit $|\Sigma| \geqq 2$ gilt $PCP_{\Gamma} \leqq_m PCP_{\Sigma}$ \\

    Beweis: \\
    Wir suchen eine effektive Transformation, die jede Instanz $I$ von $PCP_\Gamma$ in eine Instanz $I'$ von
    $PCP_\Sigma$ transformiert, sodass $I$ genau dann lösbar ist, wenn $I'$ lösbar ist. \\

    Seien $a_1,a_2 \in \Sigma$ verschieden. Seien $b_1,...,b_{|\Gamma|}$ die Elemente von $\Gamma$. \\
    Es bezeichne $\phi : \Gamma^* \rightarrow \Sigma^*$ den eindeutigen Homomorphismus von Sprachen mit
    $\phi(b_i) = a_1^i a_2$ für alle $i \in [|\Gamma|]$. \\

    Gegeben eine Instanz $I$ wie oben sei $I':=\{(\phi(u),\phi(v)):(u,v) \in I\}$ \\
    Eine Funktion, die geeignete Codes von Instanzen $I$ auf geeignete Codes der zugehörigen $I'$ abbildet
    ist berechenbar.
    
    Ist $(u_1,v_1)...(u_n,v_n)$ eine Lösung von $I$, so ist $(\phi(u_1),\phi(v_1)),...,(\phi(u_n),\phi(v_n))$
    eine Lösung von $I'$. Die Instanz $I'$ ist also lösbar, wenn $I$ lösbar ist. \\

    Ist $(u'_1,v'_1)...(u'_n,v'_n)$ eine Lösung von $I'$, so gibt es eine Folge $(u_1,v_1)...(u_n,v_n)$ von Paaren in
    $I$ mit $\phi(u_1)=u'_1$ und $\phi(v_1)=v'_1$ für alle $i \in [n]$ also mit \\
    $\phi(u_1...u_n) = u'_1...u'_n = v'_1 ...v'_n = \phi(v_1...v_n)$ \\

    Da $\phi|_\Gamma$ injektiv und $\phi(\Gamma)$ präfixfrei ist, ist $\phi$ injektiv. \\
    Folglich gilt $u_1...u_n = v_1...v_n$. $I$ ist also lösbar wenn $I'$ lösbar ist. $\square$
\end{lemma}

\begin{lemma}{}
    Für jedes Alphabet $\Sigma$ mit $|\Sigma| \geq 2$ gilt $MPCP_\Sigma \leq_m PCP_\Sigma$. \\

    Beweis: \\
    Sei $\Sigma$ ein Alphabet mit $|\Sigma| \geq 2$. \\
    Nach Lemma 4.15 genügt es ein Alphabet $\Gamma$ zu finden, sodass $MPCP_\Sigma \leq_m PCP_\Gamma$ gilt. \\

    Wir suchen eine effektive Transformation die jede Instanz $(p, I)$ von $MPCP_Sigma$ in eine Instanz
    $I'$ von $PCP_\Gamma$ für geeignete $\Gamma$ transformiert, sodass $(p,I)$ genau lösbar ist, wenn $I'$ lösbar ist. \\

    % Abb. Dominos mit Sternchen
    Betrachte die Homomorphismen von Sprachen $\sigma_{\leftarrow}, \sigma_{\rightarrow}$ $\Sigma^* \rightarrow (\Sigma \cup{*})^*$
    Für jede solche Instanz $(p,I) = ((u_1,v_1), I)$ wie oben sei $I'=\{\sigma_{\leftarrow}(u_1),*\sigma_{\rightarrow}(v_1)\}
    \cup \{\sigma_{\leftarrow}(u),*\sigma_{\rightarrow}(v)\} : (u,v) \in I \backslash\{u_1,v_1\} \cup \{**,*\}$ \\
    
    Die Funktion, die geeignete Codes von $(p,I)$ auf geeignete Codes der zugehörigen Instanzen $I'$ abbildet ist berechenbar. \\

    Gibt es eine Lösung $(u_1,v_1)...(u_n,v_n)$ von $(p,I)$, dann ist \\
    $\sigma_\leftarrow(u_1)...\sigma_\rightarrow(u_n)** = \sigma_\leftarrow(u_1...u_n)** = *\sigma_\rightarrow(u_1...u_n)*$ \\
    $*\sigma_\rightarrow(v_1)...\sigma_\rightarrow(v_n)*$ und folglich ist \\
    $(\sigma_\leftarrow(u_1),*\sigma_\rightarrow(v_1),(\sigma_\leftarrow(u_2),\sigma_\rightarrow(v_2))),...,
    (\sigma_\leftarrow(u_n),\sigma_\rightarrow(v_n)),(**,*)$ eine Lösung von $I'$. \\

    Es bleibt zu zeigen, dass $(p,I)$ lösbar ist, wenn $I$ lösbar ist. \\

    Sei $\tau : (\Sigma \cup \{*\})^* \rightarrow \Sigma^*$ der Homomorphismus von Sprachen mit $\tau|_\Sigma = id_\Sigma$
    und $\tau(*) = \lambda$. \\

    Sei $(u'_1,v'_1),...(u'_n...v'_n)$ eine Lösung von $I'$. \\
    Es gilt $\tau(u'_1)...\tau(u'_n) = \tau(u'_1...u'_n) = \tau(v'_1...v'_n)$ = $\tau(v'_1)...\tau(v'_n)$. \\
    Für $i \in [n]$ gilt $(\tau(u'_i), \tau(v'_i)) \in I \cup \{(\lambda, \lambda)\}$ \\

    Weiter gibt es mindestens ein $i \in [n]$ mit $(u'_1,v'_1) \neq (**,*)$ und damit $(u'_1,v'_1) \neq
    (\lambda, \lambda)$, denn andernfalls wäre $|u'_1...u'_n| = 2|v'_1...v'_n|$ mit $(\tau(u'_i)), \tau(v'_i)$ für alle $i \in [n]$ 
    ist eine Lösung von $I$ als Instanz von $PCP_\Sigma$ \\

    Es genügt also zu zeigen, dass $p' := (\sigma_leftarrow(u_1),*\sigma_rightarrow(v_1)) =(u'_1, v'_1). $\\
    Für $(u,v) \in I \backslash \{p',(**,*)\}$ gilt $u(1) \neq v(1)$. Somit ist $(u'_1, v'_1) \in \{p',(**,*)\}$.
    Es genügt zu zeigen, dass $(u'_1,v'_1) \neq (**,*)$ \\

    Sei $i \in \mathbb{N}$ minimal mit $(u'_1, v'_1 \neq (**,*))$ \\
    Es gilt $(v'_1,v'_n)(i) \neq *$ oder $(v'_1,...,v'_n)(i+1) \neq **$. \\
    Außerdem gilt $(**)^{i-1} * \subseteq u'_1 ... u'_n = v'_1 ... v'_n$ also folgt $i=1$,
    $2i+2+1 = 2i-1$ Sternchen $\geq i+1 \Leftrightarrow i \geq 2$, da oben an Stelle $i$ und $i+1$ ein Sternchen steht. $\square$

\end{lemma}

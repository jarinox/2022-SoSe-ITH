\vorlesung{7}{Thema der Vorlesung}{14.05.2022}

\textbf{Ziel der heutigen Vorlesung}: Zeigen, das das Postsche Korresponzendproblem über dem Binäralphabet entscheidbar ist. 
$H_{init} \leqq PCP_{0,1}$ \\

Soll aus: \\
$u(1)...(u(|u|))...v(1)...v(|v|)$ \\
$u(1)...(u(|u|))...v(1)...v(|v|)...w(1)...w(|w|)$ \\

eine Lösung werden, so muss nun $v$ oben erzeugt werden. Bei wird unten ein neues Suffix $w$ erzeugt.
Geeignete Wahl der Paare/Steine erlaubt es $w$ zu konrollieren. Werden Konfiguationen einer TM durch Wörter
dargestellt, so kann erzwungen werden, dass auf eine Darstellung einer Konfiguation die der Nachfolgekonfiguatation
folgen muss. So kann erreicht werden, dass die konstruierte Instanz genau dann lösbar ist, wenn $M$
bei Eingabe $\lambda$ terminiert. \\

Um ein größeres Alphabet zur Verfügung zu haben:

Zeige: $H_{init} \leqq_m MPCP_\Sigma \leqq_m PCP_\Sigma \leqq_m PCP_{\{0,1\}}$

%Lemma 4.15 -- gegenchecken
\begin{lemma}{}
    Für Alphabete $\Sigma$ und $\Gamma$ mit $|\Sigma| \geqq 2$ gilt $PCP_{\Gamma} \leqq_m PCP_{\Sigma}$ \\

    Beweis: \\
    Wir suchen eine effektive Transformation, die jede Insatanz $I$ von $PCP_\Gamma$ in eine Instanz $I'$ von
    $PCP_\Sigma$ transformiert, sodass $I$ genau dann lösbar ist, wenn $I'$ lösbar ist. \\

    Seien $a_1,a_2 \in \Sigma$ verschieden. Seien $b_1,...,b_{|\Gamma|}$ die Elemente von $\Gamma$. \\
    Es bezeichne $\phi : \Gamma^* \rightarrow \Sigma^*$ den eindeutigen Homomorphismus von Sprachen mit
    $\phi(b_i) = a_1a_2^i$ für alle $i \in [|\Gamma|]$. \\

    Gegeben eine Instanz $I$ wie oben sei $I':=\{(\phi(u),\phi(v)):(u,v) \in I\}$ \\
    Eine Funktion, die geeignete Codes von Instanzen $I$ auf geeignete Codes der zueghörigen $I'$ abbildet
    ist berechenbar.
    
    Ist $(u_1,v_1)...(u_n,v_n)$ eine Lösung von $I$, so ist $(\phi(u_1),\phi(v_1)),...,(\phi(u_n),\phi(v_n))$
    eine Lösung von $I'$. Die Instanz $I'$ ist also lösbar, wenn $I$ lösbar ist. \\

    Ist $(u'_1,v'_1)...(u'_n,v'_n)$ eine Lösung von $I'$, so gibt es eine Folge $(u_1,v_1)...(u_n,v_n)$ von Paaren in
    $I$ mit $\phi(u_1)=u'_1$ und $\phi(v_1)=v'_1$ für alle $i \in [n]$ also mit \\
    $\phi(u_1...u_n) = u'_1...u'_n = v'_1 ...v'_n = \phi(v_1...v_n)$ \\

    Da $\phi|_\Gamma$ injektiv und $\phi(\Gamma)$ präfixfrei ist, ist $\phi$ injektiv. \\
    Folglich gilt $u_1...u_n = v_1...v_n$. $I$ ist also lösbar wenn $I'$ lösbar ist.
\end{lemma}
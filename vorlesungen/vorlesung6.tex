\vorlesung{6}{Untentscheidbare Probleme, Halteproblem, Postsches Korrespondenzproblem}{10.05.2022}

\section*{4.1 Untentscheidbare Probleme}

\textbf{Ziel}: \\
Konstruktion von Problemen, die nicht durch TM entscheidbar sind.

\begin{defn}{diagonales Halteproblem}
    Die Menge $H_{diag} = \{e \in \mathbb{N}_0 : \Phi_e(e) \downarrow\}$ heißt das \textbf{diagonale Halteproblem}
\end{defn}

\begin{prop}{}
    Das diagonale Halteproblem ist rekursiv aufzählbar. \\
    
    \textbf{Beweis:} \\
    Die DTM, die bei Eingabe $e \in \mathbb{N}_0$ arbeitet wie $U$ bei
    der Eingabe $(e,e)$, aber bei terminierter Simulation 1 ausgibt, statt der
    Ausgabe der Simulation, berechnet die partielle charakterischte Funktion von $H_{diag}$.
    Die partielle Funktion ist somit partiell berechenbar \\
    $\Rightarrow H_{diag}$ ist rekursiv aufzählbar $\square$ \\

    \textbf{Erinnerung:} \\
    Bem. 3.12: Eine Sprache $L$ ist genau dann entscheidbar, falls $L$ und $L^c$
    rekursiv aufzählbar sind. \\

    $\Rightarrow$ Unser Plan: Wir zeigen $H_{diag}$ ist NICHT rekursiv aufzählbar.
\end{prop}

\begin{satz}{Das diagonale Halteproblem ist nicht entscheidbar}
    \textbf{Beweis:} \\
    Angenommen $H_{diag}$ wäre entscheidbar. \\
    Dann wäre die partielle charakteristische Funktion $\phi$ von $H_{diag}^c := \mathbb{N}_0 \backslash H_{diag}$
    partiell berechenbar. \\

    $\Rightarrow$ Es gibt einen Index $e$ von $\phi$. Es folgt $e \in H_{diag}^c$
    $\Leftrightarrow$ Es gibt einen Index $e$ von $\phi$. Es folgt \\
    $e \in H_{diag}^c \Leftrightarrow \phi(e)\downarrow = \Phi_e(e)\downarrow \Leftrightarrow e \in H_{diag} \Leftrightarrow e \notin H_{diag}^c \, \lightning \, \square$ \\

    \textbf{Nicht für alles gibt es einen (terminierenden) Algorithmus!}
\end{satz}

\textbf{Nächster Schritt:} \\
Finden von weiteren nicht entscheidbaren Problemen durch Reduktion auf diag. Halteproblem.

\begin{defn}{m-Reduktion}
    Gegeben eine Sprache $A$ über dem Alphabet $\Sigma$ und eine Sprache $B$ über einem Alphabet $\Gamma$. \\
    Dann heißt $A$ \textbf{many-to-one-reduzierbar} (auch m-reduzierbar) auf $B$, fall es eine berechenbare Funktion
    $f : \Sigma^* \rightarrow \Gamma^*$ gibt, sodass: \\
    $w \in A \Leftrightarrow f(w) \in B \, \forall w \in \Sigma^*. $\\
    In solch einem Fall schreiben wir $A \leq_m B$. \\

    Falls $A \leq_m B$ und $B \leq_m A$, so sind $A$ und $B$ \textbf{many-to-many-äquivalent}, kurz m-äquivalent (auch $a =_m B$). \\

    \textbf{Intuitiv:} $A \leq_m B$ bedeutet, mind. so schwer wie $A$.
\end{defn}

\begin{bem}
    \begin{itemize}
        \item "$\leq_m$ "{} ist transitiv (Verkettung)
        \item Ist $A \leq_m B$ und ist $B$ entscheidbar, so ist $A$ entscheidbar.
        \item Alle entscheidbaren Sprachen $L$ mit $\emptyset \neq L \neq \mathbb{N}_0$ sind m-äquivalent
    \end{itemize}
\end{bem}

\textbf{Nächster Schritt:} \\
Das initiale Halteproblem ist nicht endscheidbar. \\

Wir definieren $H_{init}\{e \in \mathbb{N}_0 : \Phi_e(0)\downarrow\}$ \\

Unser Plan ist zu zeigen, dass $H_{diag} \leq_m H_{init}$ gilt.
Dazu suchen wir eine berechenbare Funktion $f: \mathbb{N}_0 \rightarrow \mathbb{N}_0$, sodass $\Phi_e(e)\downarrow \Leftrightarrow \Phi_{f(e)}(0)\downarrow$.

\begin{satz}{Das init. Halteproblem ist nicht entscheidbar}
    \textbf{Beweis:}\\
    Sei $\psi : \mathbb{N}_0^2 \rightarrow \mathbb{N}_0$ mit $\psi(e,x) = \Phi_e(e) \, \forall e,x \in \mathbb{N}_0$. Dann ist $\psi$ partiell berechenbar. \\
    Sei $e_0$ ein Index von $\psi$ und sei $s:\mathbb{N}_0^2 \rightarrow \mathbb{N}$ gemäß dem $s^m_n$-Theorem so gewählt, 
    dass \\ $\Phi_{e_0}(e,x) = \Phi_{s(e_0,e)}(x) \, \forall e,x \in \mathbb{N}_0$ \\
    Sei $f: \mathbb{N}_0 \rightarrow \mathbb{N}_0$ mit $f(e) = s(e_0,e) \, \forall e\in \mathbb{N}_0$ \\

    Dann ist $f$ berechenbar und für alle $e \in \mathbb{N}_0$ gilt:\\
    $e \in H_{diag} \Leftrightarrow \Phi_e(e)\downarrow \Leftrightarrow \psi(e,0)\downarrow \Leftrightarrow \Phi_{e_0}(e,0)\downarrow \\
    \Leftrightarrow \Phi_{s(e,e_0)}(0)\downarrow \Leftrightarrow \Phi_{f(e)}(0)\downarrow \Leftrightarrow f(e) \in H_{init}$ \\

    $\Rightarrow H_{diag} \leq_m H_{init}$, da $H_{diag}$ nicht entscheidbar ist, ist $H_{init}$ auch nicht entscheidbar. (Bem 4.12). $\square$
\end{satz}

\begin{defn}{(modifiziertes) Postsches Korrespondenzproblem, Emil Post}
    Für ein Alphabet $\Sigma$ sei eine \textbf{Instanz} des (modifizierten) Postschen Korrespondenzproblems über $\Sigma$ eine Teilmenge $I \subseteq (\Sigma^*)^2$ 
    (ein Paar $(p, I)$ mit $I \subseteq (\Sigma^*)^2$ und $p \in I$). \\

    Eine \textbf{Lösung} für eine Instanz ist eine endliche Folge $(u_1,v_1)...(u_n,v_n)$ von Paaren aus $I$ (mit $p \in (u_1,v_1)$) mit $n \geq 1$, sodass
    $u_1 \circ ... \circ u_n = v_1...v_n$ gilt. Gibt es eine Lösung für eine Instanz $I$, so heißt die Instanz \textbf{lösbar}. Das (modifizierte) Postsche 
    Korrespondenzproblem über $\Sigma$, kurz $PCP_\Sigma (MCPC_\Sigma)$ ist die Menge aller lösbaren Insanzen (modifizierter) Postscher Korrespondenzprobleme 
    über $\Sigma$. \\

    Für ein festes Alphabet $\Sigma$ lassen sich die Instanzen leicht in geeigneter Weise effektiv als Binärwort codierten, sodass $PCP_\Sigma$ bzw. $MPCP_\Sigma$ 
    als Sprache aufgefasst werden kann.
\end{defn}

\textbf{Ziel:} \\
Wir zeigen, dass das Postsche Korrespondenzproblem über $\Sigma$ mit $|\Sigma| \geq 2$ nicht entscheidbar ist. \\

\textbf{Idee:} TM-Simulation: \\
\begin{align*}
    |w_1| \geq |w_2| && \dfrac{w_1}{w_2} = \dfrac{w_2 \circ w_3}{w_2} \\
\end{align*}

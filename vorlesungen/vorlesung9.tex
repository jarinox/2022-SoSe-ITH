\vorlesung{9}{Fixpunktsatz, Rekursionstheorem und Satz von Rice}{19.05.2022}

%%4.2 - wo 4.1?
\subsection{Fixpunktsatz, Rekursionstheorem und Satz von Rice}
 
\textbf{Letztes Kapitel:} Es gibt unentscheidbare Probleme. \\
(Beweis durch Standardaufzählung der TM, Spalte = TM, Zeile = Eingabe) \\

\textbf{Ziel dieses Kapitels:} Weitere Konsequenzen der Standardaufzählung. \\

\textbf{Programmtransformationen:} Abbildungen von (normierten) TM auf TM. \\

Jede TM hat einen Index $e \in \mathbb{N}_0$ \\
Transformation von TM auf TM $\Leftrightarrow f: \mathbb{N}_0 \rightarrow \mathbb{N}_0$ \\
$\Phi_{f(e)}$
$\Phi_{\Phi_{e}}(x)$ falls $\Phi_e(x)\uparrow$ ist $\Phi_{\Phi_e(x)}$ die 
part. berechenbare Funktion mit leerem Definitionsbereich \\

Dazu passend als Schema: \\
$\Phi_{\Phi_0(0)} \, \Phi_{\Phi_0(1)} \, \Phi_{\Phi_0(2)}...$ \\
$\Phi_{\Phi_1(0)} \, \Phi_{\Phi_1(1)} \, \Phi_{\Phi_1(2)}...$ \\
$...$ \\

Sei $\eta : \mathbb{N}_0 \leadsto \mathbb{N}_0$ die part. berech. part. Funktion, sodass 
$\eta = \Phi_e(e) \, \forall e \in \mathbb{N}_0$ \\
$\Rightarrow \Phi_{\Phi_e(e)} = \Phi_{\eta(e)} \, \forall e \in \mathbb{N}_0$ \\

Sei $f: \mathbb{N}_0 \rightarrow \mathbb{N}_0$ eine berechb. Funktion, so gilt obiges insbesonderen für den Index $e'$
der Funktion $f \circ \eta$ \\

Es gilt: \\
$\Phi_{f(\eta(e'))} = \Phi_{\Phi_{e'}(e')} = \Phi_{\eta(e')}$. \\

Fassen wir das als effektive Programmtransformation auf und wäre $\eta(e')\downarrow$, so wäre $\eta(e')$ in
gewissem Sinne ein Fixpunkt von $f$, nämlich $M_{\eta(e')}$ wird auf eine TM abgebildet, die die gleiche part. Funktion berechnet.

\begin{defn}{Fixpunkt}
    Ein \textbf{Fixpunkt} einer berechenbaren Funktion $f: \mathbb{N}_0 \rightarrow \mathbb{N}_0$, ist ein 
    $e \in \mathbb{N}_0$ mit $\Phi_{f(e)} = \Phi(e)$.
\end{defn}

\begin{satz}{Fixpunktsatz, 1967, Rogers}
    Alle berechenbaren Funktionen $f: \mathbb{N}_0 \rightarrow \mathbb{N}_0$ haben einen Fixpunkt. \\

    \textbf{Beweis:} \\
    Für $e,x \in \mathbb{N}_0$ mit $\Phi_e(x)\downarrow$, sei $\phi_{e,x} : \mathbb{N}_0 \rightarrow \mathbb{N}_0$ die 
    part. berechenbare part. Funktion mit $\phi_{e,x} = \Phi_{\Phi_e(x)}(y) \, \forall y \in \mathbb{N}_0$. \\
    Fall $e,x \in \mathbb{N}_0$ und $\Phi_e(x)\uparrow$ sei $\phi_{e,x}: \mathbb{N}_0 \leadsto \mathbb{N}$ die part. berechenbare part.
    Funktion mit $dom(\phi_{e,x}) \neq \emptyset$ \\

    Sei $\tau : \mathbb{N}_0^2 \leadsto \mathbb{N}_0$ die part. berechenbare Funktion mit  $\tau_{e,x} = \Phi_{Phi_e(e)}(x) \,
    \forall e,x \in \mathbb{N}_0$. \\

    Sei $e_\tau$ ein Index von $\phi \Rightarrow$ ($S^m_n$-Theorem) es existiert eine berechenbare Funktion
    $s'_1: \mathbb{N}_0^2 \rightarrow \mathbb{N}$ mit \\
    $\Phi_{s'_1(e_\tau,e)}(x) = \tau(e,x) \, \forall e,x \in \mathbb{N}_0$ \\

    Sei $h: \mathbb{N}_0 \rightarrow \mathbb{N}_0$ die berechenbare Funktion $h(e) = s'_1(e_\tau,e) \, \forall e \in \mathbb{N}_0$ \\
    Dann gilt $\Phi_{h(e)}(x) = \Phi_{s'_1(e_\tau,e)}(x) = \tau(e,x) = \Phi_{\Phi_e(e)}(x)$, also \\
    $\Phi_{h(e)} = \Phi_{\Phi_e(e)}(x)$ \\

    Sei $e_{f \circ h}$ ein Index der Funktion $f \circ h$ und $e_{\text{fix}}:=h(e_{f \circ h})$. \\
    Somit gilt: \\
    $\Phi_{f(e_{\text{fix}})} = \Phi_{f(h(e_{\text{fix}}))} = \Phi_{\Phi_{e_{f \circ h}}}(e_{f \circ h}) = \Phi_{h(e_{f \circ h})} = \Phi_{e_{\text{fix}}}$ \\

    Folglich ist $e_{fix}$ ein Fixpunkt von $f$. $\square$
\end{satz}

\textbf{Wichtig:} Die Fixpunkte wie oben definiert sind \textbf{semantische} Fixpunkte aber \textbf{nicht syntaktische}-
(Das Programm macht das Gleiche, aber ist im Syntax möglicherweise verschieden.)

\begin{satz}{Rekursionstheorem, Kleene, 1938}
    Für alle partiell berechenbaren part. Funktionen $\phi: \mathbb{N}_0^2 \leadsto \mathbb{N}_0$ gibt es ein $e$ mit 
    $\Phi_e(x) = \phi_{(e,x)} \, \forall x \in \mathbb{N}_0$ \\

    \textbf{Beweis:} \\
    Sei $e_\phi$ ein index von $\phi$. Gemäß $s^n_m$-Theorem, gibt es eine berechenbare Funktion $s'_1 : \mathbb{N}_0^2 \rightarrow \mathbb{N}$ 
    mit $\Phi_{s'_1(e_\phi,e)}(x) = \phi(e,x) \, \forall x \in \mathbb{N}_0$. \\

    Sei $s: \mathbb{N}_0 \rightarrow \mathbb{N}_0$ die berechenbare Funktion $s(e)= s'_1(e_\phi,e) \, \forall e \in \mathbb{N}_0$
    $\Rightarrow$ (Fixpunktsatz) s hat einen Fixpunkt $e_{text{fix}}$ mit \\
    $\Phi_{e_{\text{fix}}}(x) = \Phi_{s(e_{\text{fix}})}(x) = \Phi_{s'_1(e_\phi,e_{\text{fix}})}(x) = \phi(e_{\text{fix}}) \, \forall x \in \mathbb{N}_0$. $\square$
\end{satz}

\begin{kollar}{}
    Es gibt ein $e \in \mathbb{N}_0$ mit $\Phi_e(x) =e$. \\

    \textbf{Beweis:}\\
    Sei $tau : \mathbb{N}_0^2 \leadsto \mathbb{N}_0$ die part. berechenbare part. Funktion mit $\phi(e,x) = e$ für alle $e,x \in \mathbb{N}_0$ \\
    $\Rightarrow$ (Satz 4.22) $\exists e \in \mathbb{N}_0$ mit $\Phi_e(x) = \tau(e,x)=e \, \forall x \in \mathbb{N}_0$ \\

    \textbf{Bemerkung:} \\
    Es gibt Programme, die ohne auf den Speicher des Computers zugreifen, ihren eigenen Code ausgeben. (Quines)
\end{kollar}

Eine weitere wichtige Konsequenz aus dem Fixpunktsatz ist die Einsicht, dass nicht triviale semantische Programmeigenschaften unentscheidbar sind. \\
Jede Menge $\emptyset \neq I \subsetneq \mathbb{N}_0$, die in dem Sinne durch eine semantische Programmeigenschaft charakterisiert ist, dass
$e \in I \Leftrightarrow e' \in I \, \forall e,e' \in \mathbb{N}_0$ mit $\Phi_e = \Phi_{e'}$ gilt, ist unentscheidbar. \\

\begin{defn}{Indexmenge}
    Eine Teilmenge $I \subseteq \mathbb{N}_0$ heißt Indexmenge, wenn $e \in I \Leftrightarrow e' \in I \, \forall e,e' \in \mathbb{N}_0$
    mit $\Phi_e = \Phi_{e'}$ gilt.
\end{defn}

\begin{satz}{Satz von Rice, 1951}
    Ist $I$ eine Indexmenge mit $\emptyset \neq I \neq \mathbb{N}_0$, so ist $I$ nicht entscheidbar (keine TM mit akzeptierter Sprache). \\

    \textbf{Beweis:} \\
    Sei $e_0 \notin I$, $e_1 \in I$ und sei $f: \mathbb{N}_0 \rightarrow \mathbb{N}_0$ die Funktion mit 
    $f(e) = e_1 \, \forall e \in \mathbb{N}_0 \backslash I$ und $f(e) = e_0 \, \forall e \in I$. \\

    $\forall e \in \mathbb{N}_0$ gilt $f(e) \in I \Leftrightarrow e \notin I$ und da $I$ eine Indexmenge ist somit 
    $\Phi_{f(e)} \neq \Phi_e$. Die Funktion $f$ hat also keinen Fixpunkt. \\
    
    Wäre $I$ entscheidbar, so wäre $f$ berechenbar, was im Widerspruch zum Fixpunktsatz steht.
    $\square$
\end{satz}

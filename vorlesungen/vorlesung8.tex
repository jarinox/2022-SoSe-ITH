\vorlesung{8}{Thema der Vorlesung}{17.05.2022}

\begin{lemma}{}
    Es gilt $H_{init} \leq_m MCPC_{\{\square, 0,1\},*,\#,\dagger}$ \\

    \textbf{Beweisskizze:} \\
    Wir suchen eine effektive Transformation, die jede natürliche Zahl (Nummer der Turingmaschine) $e$, auf eine 
    Instanz $(p_e,I_e)$ von $MCPC_{\{\square, 0,1\},*,\#,\dagger}$ abbildet, sodass $M_e(\lambda)\downarrow$ genau dann gilt,
    falls $(p_e,I_e)$ lösbar ist. \\

    Sei $e \in \mathbb{N}_0$. Sei $Q$ die Zustandsmenge und $\Delta$ die Übergangsregel von $M_e$.
    Es gilt als $M_e=(Q,\Sigma, \Gamma, \Delta, s, F)$ für $\Sigma = \{0,1\}, \Gamma = \{\square,0,1\}, s=0$ und $F=\{0\}$ \\

    %%Abbildung8.1%%
    
    Für eine Instanz $(p,I)$ des modizifizierten PCP über einem Alphabet bezeichnet die Folge 
    $p = (u_1,v_1)...(u_n,v_n)$ für die $u_1,...,u_n \subseteq v_1,...v_n$ (oder umgekehrt) eine partielle Lösung von
    $(p,I)$. Wir wollen $(p_e,I_e)$ so wählen, dass partielle Lösungen von $(p_e,I_e)$ partiellen Rechenungen von $M_e$ entsprechen. 
    Dabei codieren wir eine Konfiguation $(q,w,p) \in Q \times \Gamma^* \times \mathbb{N}_0$ von $M_e$ durch das Wort
    $code(q,w,p) := \#w(1)...w(p-1)*bin(q)*w(p)...w(\vert w \vert)\#$. \\

    Im wesentlichen wollen wir erreichen, dass es genau dann ein Wort $w$ eine partielle Lösung $(u_1,v_1)...(u_n,v_n)$
    von $(p_e,I_e)$ mit $w = v_1...v_n$ gilt, $w$ Präfix der Konkatination von $code(c_1)...code(c_n)$ der Codes von Konfigurationen
    einer partiellen Rechnung $c_1,...,c_n$ von $M_e$ bei Eingabe $\lambda$ ist

    Eine partielle Lösung soll ganu dann zu einer Lösung von $(p_e,I_e)$ vervollständigt werden können, falls die durch $w$ berschriebene 
    partielle Rechnung mit einer Stoppkonfiguration endet, also eine endl. Rechnung ist. Dann ist $(p_e, I_e)$ genau dann lösbar, wenn 
    die Rechnung von $M_e$ zur Eingabe $\lambda$ endlich ist. \\

    Für $q \in Q$ schreiben $\hat{q} := * bin(q) *.$ \\
    Startpaar nehmen wir $p_e = (0,0,\# \hat{s} \square \#)$. \\
    Die Elemente von $I_e$ sind die folgenden:
    
    %%Abbildung 8.2
    
    \begin{itemize}
        \item $p_e$
        \item Für das kopieren von nicht veränderten Präfixen un Suffixen von Konfig: $(\#,\#),(\square,\square),(0,0),(1,1)$
        \item Für Schritte nach links $(\square \hat{q} a, \hat{q}' \square a), (0\hat{q} a, \hat{q} 0 a'), (1 \hat{q} a, \hat{q}' 1 a'), (\# \hat{q} a, \# \square \hat{q}'a')$
              für jede Instruktion $(q,a,q',a',L) \in \Delta$
        \item Für Instruktionen mit Bewegung S (stehenbleiben): $(\hat{q} a, \hat{q}',a') \, \forall (q,a,q',a',S) \in \Delta $
        \item Für Schrite nach rechts: $(\hat{q} a \square, a' \hat{q}' \square),(\hat{q} a 0, \hat{q}' a'0),(\hat{q} a 1, \hat{q}' a' 1), \\
        (\hat{q} a \#, a' \hat{q}' \square \#) \, \forall (q,a,q',a',R) \in \Delta$
        \item Für den Beginn der Vervollständigung zu einer Lösung, wenn eine Stoppkonfiguration erreicht wurde: \\
              $(\hat{q} a, \dagger a) \, \forall q \in Q, n \in \Gamma$ für die $(q,a)$ nicht Bedingungsteil einer Instriktion in $\Delta$ ist.
        \item Für den Abbau von Konfigurationen: $(\dagger 0, \dagger),(\dagger \square, \dagger),(\dagger1,\dagger)$ und $(0\dagger,\dagger),
              (1 \dagger,\dagger),(\square \dagger,\dagger)$
        \item Für den Abschluss $(\#\dagger\#0,0)$
    \end{itemize}

    %%Abbsildung 8.3
\end{lemma}

\begin{satz}{}
    Für jedes Alpabet $\Sigma$ mit $\vert \Sigma \vert \geq 2$ ist $PCP_\Sigma$ nicht entscheidbar. \\
    \textbf{Beweis}: Mit Lemma 4.15, 4.16, 4.17 folgt $H_{init} \leq_m MPCP_{\{\square, 0,1,*,\dagger,\#\}} \leq_m PCP_{\{\square,0,1,*,\dagger,\#\}}
    \leq_m PCP_{Sigma}$.\\
    $\Leftrightarrow PCP_\Sigma$ ist nicht entscheidbar, da $H_{init}$ nicht entscheidbar ist. $\square$
\end{satz}
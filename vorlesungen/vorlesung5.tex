\vorlesung{5}{Berechenbarkeit}{05.05.2022}

\section{Berechenbarkeit}

\textbf{Nächster Schritt}: \\
Codierung von TM durch Zahlen (bspw. Binärwörter). Das ist quasi unser Quellcode für TM. \\
Im Folgenden verwenden wir oft natürliche Zahlen $e \in \mathbb{N}$ oder $(e_1,...,e_n) \in \mathbb{N}$ 
für ein $n \in \mathbb{N}$ als Eingaben für TM, was bedeutet, dass wir als Eingabe $bin(e)$ 
bzw. $(bin(e_1),...,bin(e_n))$ nehmen. \\

Insbesondere erlaubt es uns partiell berechenbare partielle Funktionen als partielle Funktionen von $\mathbb{N}^n \leadsto \mathbb{N}$.
Zudem können Sprachen $L \subseteq \{0,1\}^*$ auch als Teilmenge von $\mathbb{N}$ auffassen.

\begin{defn}{Code}
    Wir betrachten eine Funktion $code$ (mit geeignetem Definitionsbereich) und Zielmenge $\{0,1\}^*$, sodass folgendes gilt:
    \begin{itemize}
        \item Zunächst sei $code(L)=10$, $code(S)=00$, $code(R)=01$
        \item Für eine Instruktion $I = (q,a,q',a',B) \in \mathbb{N}_0 \times \{0,1\} \times \mathbb{N}_0 \times \{0,1\} \times \{L,S,R\}$
              einer normierten TM sei \\
              $code(I) := 0^{| bin(q) |} \, 1bin(q) \, a \, 0^{| bin(q')|} \, 1bin(q') \, a' \, code(B)$
        \item Für eine endliche Menge $\Delta \in \mathbb{N}_0 \times \{0,1\} \times \mathbb{N}_0 \times \{0,1\} \times \{L,S,R\}$ von Instuktionen einer
              normierten TM und $i \in [| \Delta |]$ sei $code_i(\Delta)$ das in längenlex. Ordnung i-te Wort in 
              $code(I) : I \in \Delta$ und es sei $code(\Delta) = code_1(\Delta),...,code_{| \Delta |}(\Delta)$
        \item Für eine TM $M= (\{0,1,...,n\},\{0,1\},\{0,1,\square\},\square,0,\{0\})$ sei \\
              $code(M) := 0^{| bin(n) |} \, 1bin(n) \, code(\Delta)$ 
    \end{itemize}
\end{defn}

Die Details der Codierung von TM sind weitgehend unwichtig. Relevant ist, dass es eine effektive Codierung 
von TM durch Binärwörter $w \in \{0,1\}^*$ gibt, sodass folgendes gilt:
\begin{itemize}
    \item Jede normierte TM hat einen Code.
    \item Die Sprache der Codes von TM ist entscheidbar. (Nicht trivial)
    \item Codes können in geeignete Repräsentationen der durch sie codierten TM umgewandelt werden, die es 
          insbesondere erlauben die codierten TM effektiv zu simulieren.
    \item Geeignete Repräsentationen von TM könnten effektiv in ihre Codes umgewandelt werden.
\end{itemize}

\begin{defn}{Standardaufzählung}
    Sei $\hat{w}_0, \hat{w}_1,...$ die Aufzählung aller Codes normierten DTM in längenlex. Ordnung. Für $e \in \mathbb{N}_0$ sei $M_e$
    die durch $\hat{w}_e$ codierte DTM und für $n \in \mathbb{N}$ sei $\Phi_e^n : \mathbb{N}_0^n \leadsto \mathbb{N}_0$ die von $M_e$ 
    berechnete n-äre partielle Funktion.\\
    
    Für $n \in \mathbb{N}$ sei $(\Phi_e^n)_{e \in \mathbb{N}_0}$ die \textbf{Standardaufzählung} der n-ären partiell berechenbaren part. Fkt. 
    Für $n \in \mathbb{N}$ und eine part. berechenb. partielle Funktion $\phi$
    $\phi : \mathbb{N}_0^n \leadsto \mathbb{N}_0$ heißt jede natürliche Zahl $e \in \mathbb{N}_0$ mit $\Phi_e^n = \phi$ Index von $\phi$. \\
    Ist $n$ klar aus dem Kontext, so schreiben wir auch $\Phi_e$ anstatt $\Phi_e^n$. \\
\end{defn}

\begin{bem}
    Für $n \in \mathbb{N}$ und jede partiell berechenbare Funktion $\phi : \mathbb{N}_0^n \leadsto \mathbb{N}_0$ gibt es \textbf{unendlich} viele Indices von $\phi$.
\end{bem}

\begin{defn}{$U$}
    Es bezeichne $U$ die normierte DTM, die bei Eingabe $(e,x_1,...,x_n) \in \mathbb{N}_0^{n+1}$, wobei $n \in \mathbb{N}$ 
    die normierte DTM $M_e$ bei Eingabe $(x_1,...,x_n)$ simuliert und falls diese terminiert die Ausgabe der Simulation ausgibt.
\end{defn}

\begin{defn}{universell}
    Eine DTM heißt \textbf{universell}, wenn es für alle $n \in \mathbb{N}$ und alle part. berechneten 
    partiellen Funktionen $\Phi : \mathbb{N}_0^n \leadsto \mathbb{N}_0$ ein $e \in \mathbb{N}_0$, sodass
    $U(e,x_1,...,x_n) = \phi(x_1,...,x_n)$ für alle $x_1,...,x_n \in \mathbb{N}_0$ gilt. 
\end{defn}

\begin{bem}
    Die DTM $U$ ist universell, denn für $e \in \mathbb{N_0}, n \in \mathbb{N}$ und $x_1,...,x_n \in \mathbb{N}_0$ gilt: \\
    $\phi_U(e,x_1,...,x_n) = \Phi_e(x1,...,x_n)$. \\

    Anschaulich spielen universelle TM (insbesondere $X$) die Rolle von Computern.
\end{bem}

In formalen Konstruktionen ist es oft hilfreich, dass es für jede (n+1)-äre partielle Funktion effektiv in eine n-äre
Funktion umgewandelt werden kann.

\begin{satz}{$S^m_n$-Theorem, Kleene}
    Für alle $n,m \in \mathbb{N}$ eine \textbf{berechenbare Funktion} $S^m_n : \mathbb{N}_0^{m+1} \rightarrow \mathbb{N}_0$ mit \\
    $\Phi^{m+n}_e(x_1,...,x_m,y_1,...,y_n) = \Phi_{s^m_n(e,x_1,x_n)}(y_1,...y_n)$ für alle $e,x_1,...,x_m,y_1,...y_n$ \\

    \textbf{Beweis}:\\
    Fixiere $m,n$. \\
    Betrachte die DTM $S$, die bei Eingabe $(e,x_1,...,x_n) \in \mathbb{N}^{m+1}_0$ wie folgt verfährt:
    \begin{itemize}
        \item Zunächst bestimmt $S$ den Code von $M_e$.
        \item Der Code von $M_e$ wir dann in einem Code einer normierten TM umgewandelt,
              die zunächst $x_1 \square x_2 \square ... \square x_m \square$ links neben
              die Eingabe schreibt und den Kopf auf das erste Feld links neben dem beschriebenen
              Bandteil bewegt und dann wie $M_e$ arbeitet.
        \item Der Code von $M$ wird ausgegeben und bestimmt an welcher Stelle der Standardaufzählung
              der Code auftaucht.
    \end{itemize}
    Sei $s^m_n$ die von $S$ berechnete (m+1)-äre Funktion. Dann ist $s^m_n$ wie gewünscht. $\square$
\end{satz}
